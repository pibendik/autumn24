\documentclass{article}
\usepackage[utf8]{inputenc}

\usepackage{bussproofs}
\title{Oblig 1}
\author{Bendik Svalastog}
\date{}
% Logical connectives
\newcommand{\impl}{\rightarrow}
\newcommand{\mand}{\wedge}
\newcommand{\mor}{\vee}

% Shorthand for common proof constructs
\def\seq{\Longrightarrow}
\def\fCenter{\; \seq \;}
\newcommand{\scr}{\scriptsize}
\newcommand{\fns}{\footnotesize}
% To label rules. Example: \rlabel{\neg} 
%                       => \RightLabel{$\neg$-\footnotesize{right}}
%                          \llabel{\forall}
%                       => \RightLabel{$\forall$-\footnotesize{left}}
\newcommand{\rlabel}[1]{\RightLabel{$#1$-\fns{right}}}
\newcommand{\llabel}[1]{\RightLabel{$#1$-\fns{left}}}
\newcommand{\ax}{\RightLabel{\fns{axiom}}}

\begin{document}

\maketitle

\section*{Exercise 1}

This is my attempt at "Oblig 1".

\subsection*{a)}

I was unsure about the implication left rule. It feels like I misunderstood something. If I find the time for it before the deadline, I'll enter into latex what I was thinking.
I also believe I might have made mistakes in latex. It is very confusing to write all these "backslash mand" and "backslash fCenter" things. And lastly, I think I missed something about what we typically do when we arrive at a place where the same variable appears twice, as in the first formula with "p, q, p" in the antecedent. I think I remember hearing that we close that branch where that happens, but I couldn't find it explicitly in the Ben-Ari book.


\subsubsection*{$F1$}

...Anyway, following is the proof for $\fCenter ((p \impl q) \mand (q \impl r)) \impl (p \impl r)$:


\begin{prooftree}


\AxiomC{~}
\ax
\UnaryInfC{$p, q \fCenter p, r $}

% Right branch
\AxiomC{~}
\ax
\UnaryInfC{$ p \fCenter p, q, r$}
\llabel{\impl}
\BinaryInfC{$ p, (p \impl q) \fCenter r, q$}

\AxiomC{~}
\ax
\UnaryInfC{$p, r \fCenter p, r $}
% Right branch
\AxiomC{~}
\ax
\UnaryInfC{$ p, r, q \fCenter r$}

\llabel{\impl}
\BinaryInfC{$ p, r, (p \impl q) \fCenter r$}
% BinaryInfC creates two branches
\llabel{\impl}
\BinaryInfC{$ p, (p \impl q), (q \impl r) \fCenter r$}
\llabel{\mand}

\UnaryInfC{$ p, ((p \impl q) \mand (q \impl r)) \fCenter r$}
\rlabel{\impl}
\UnaryInfC{$ ((p \impl q) \mand (q \impl r)) \fCenter (p \impl r)$}
% Adjusted bottom line
\rlabel{\impl}
\UnaryInfC{$\fCenter ((p \impl q) \mand (q \impl r)) \impl (p \impl r)$}
\end{prooftree}


\subsubsection*{$F2$}


I did not find time to put this in latex, so I added it as a photo of my attempt in the delivery.
I'm allmost certain I misunderstood something about choosing constants and variables that "allready exist" earlier or other places in the tree. I intend to brush up on this.

\begin{prooftree}
% Left branch
\AxiomC{~}
\ax
\UnaryInfC{$ \forall x \exists y ( p(x) \land (p(y) \rightarrow q(x)) ) \fCenter \forall z q(z)$}
\rlabel{\impl}
\UnaryInfC{$\fCenter \forall x \exists y ( p(x) \land (p(y) \rightarrow q(x)) ) \rightarrow \forall z q(z)$}
\end{prooftree}

\subsubsection*{$F3$}
I have a question here. I had two implications after the $\mand$ -left, but I don't know that I logically had to work through both? Like, did the order in which I branch out matter?  I checked on paper, and got the same type of non-closed branches if I chose the other order. Also, I kept repeat terms like "p, q, q" because I wanted to show how I had worked through things. I hope that in it self won't make the proof wrong.

As you can see, I ended up with branches that did not close, and so I believe this to be a counter model (showing that $F3$ is invalid using the sequent calculus)

Proof:
\begin{prooftree}
  % Left branch
  \AxiomC{$p, r \fCenter q, q$}

  \AxiomC{}
  \UnaryInfC{$p \fCenter p, q, q$}

  \BinaryInfC{$p, p \rightarrow r \fCenter q, q$}

  
  \AxiomC{$p, r, r, \fCenter q$}

  \AxiomC{}
  \UnaryInfC{$p, r\fCenter p, q$}

  \BinaryInfC{$p, r, p \rightarrow r \fCenter q $}
  \rlabel{\rightarrow}
  
  \BinaryInfC{$p, p \rightarrow r, q \rightarrow r \fCenter q$} %first branch
  \llabel{\mand}
  \UnaryInfC{$p, (p \rightarrow r) \mand ( q \rightarrow r ) \fCenter q$}
  \rlabel{\rightarrow}  
  \UnaryInfC{$(p \rightarrow r) \land (q \rightarrow r) \fCenter p \rightarrow q$}
  \rlabel{\rightarrow}
  \UnaryInfC{$\fCenter ((p \rightarrow r) \land (q \rightarrow r)) \rightarrow (p \rightarrow q)$}
\end{prooftree}


\newpage

\section*{Exercise O1.2}
\subsection*{a)}
\subsubsection*{Bendik's $\uparrow$-left rule}

\AxiomC{$\Gamma, A, B \fCenter \Delta$}
\rlabel{\uparrow}
\UnaryInfC{$\Gamma \fCenter (A \uparrow B), \, \Delta$}
\DisplayProof

\subsubsection*{Bendik's $\uparrow$-right rule}


\AxiomC{$\Gamma \fCenter A , \Delta$}
\AxiomC{$\Gamma \fCenter B , \Delta$}
\rlabel{\uparrow}
\BinaryInfC{$\Gamma, (A \uparrow B) \fCenter \Delta$}
\DisplayProof

\subsection*{b)}
The property that is shown for each of the rules in the soundness proof for propositional LK, where a property is shown separately for each of the rules of the calculus, is preserving falsifiability upward.

I have a picture of what I ended up with when I tried to show that my rules preserve falsifiability. I'll add it too to Devilry.


\end{document}
